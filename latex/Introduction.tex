\subsection{Zahlbereichserweiterung}

\parbox{0.9\textwidth}{
    Ziel ist es Gleichungen wie $\mathbf{x^2 + 1 = 0}$ zu lösen.
    Man definiert dazu nun eine Einheit $\mathbf{i}$ mit der Eigenschaft $\mathbf{i^2 = -1}$.
    Für so eine Zahl ist nun jedoch kein Platz mehr in der Zahlengerade, 
    deshalb erweitert man die Zahlengerade mit den Reelen Zahlen zu einer \textbf{Zahlenebene}.
    Somit sind die komplexen Zahlen eine \textbf{Erweiterung der Reellen Zahlen}.
    Dies wird auch \textbf{Zahlbereichserweiterung} genannt
}

\begin{tikzpicture}
    \begin{axis}[
        samples=100,
        axis lines=center,
        x=1cm,
        y=1cm,
        xmin=-5, xmax=5,
        ymin=-5, ymax=5,
        xtick={-4, -3,...,4},
        ytick={-4, -3,...,4},
        xticklabels={$-5$,$-4$,$-3$,$-2$,$-1$,$1$,$2$,$3$,$4$,$5$},
        yticklabels={$-5i$,$-4i$,$-3i$,$-2i$,$-1i$,$1i$,$2i$,$3i$,$4i$,$5i$},
        xlabel=$\mathbb{R}$,
        ylabel=$\mathbb{C}$
    ]
    
    \draw[->] (0, 0) -- (2, 3) node[above right]{a + bi};

    
    \draw[decorate, decoration = {brace,amplitude=6pt,raise=1pt}] (0, 3) -- (2, 3) node[midway, above, yshift=5pt]{a};
    \draw[decorate, decoration = {brace,amplitude=6pt,raise=1pt,mirror}] (2, 0) -- (2, 3) node[midway, right, xshift=5pt]{b};

    \end{axis}
\end{tikzpicture}