\subsection{Zahlbereichserweiterung}

\parbox{0.9\textwidth}{
    Ziel ist es Gleichungen wie $\mathbf{x^2 + 1 = 0}$ zu lösen.
    Man definiert dazu nun eine Einheit $\mathbf{i}$ mit der Eigenschaft $\mathbf{i^2 = -1}$.
    Für so eine Zahl ist nun jedoch kein Platz mehr in der Zahlengerade, 
    deshalb erweitert man die Zahlengerade mit den Komplexen Zahlen zu einer \textbf{Zahlenebene}.
    Somit sind die komplexen Zahlen eine \textbf{Erweiterung der Reellen Zahlen}.
    Dies wird auch \textbf{Zahl(en)bereichserweiterung} genannt
} 

\subsection{Einführung}

\parbox{0.9\textwidth}
{
    Eine komplexe Zahl $\mathbf{z \in  \mathbb{C}}$ ist \textbf{ein Wertepaar} $\mathbf{(a; b)}$, wobei $\mathbf{a}$ den \textbf{Realenteil} und $\mathbf{b}$ den \textbf{Imaginärteil} darstellt.

    \begin{tabular}{c|c}
        \parbox{0.45\textwidth}
        {
            \begin{alignat*}{3}
                &z &&= \left(a; b\right) \qquad z \in \mathbb{C} \\ 
                &a &&= \mbox{Re}(z) \\ 
                &b &&= \mbox{Im}(z) \\
            \end{alignat*}
        }
        &
        \parbox{0.45\textwidth}
        {
            \begin{alignat*}{3}
                &\left(a; b\right) + \left(c; d\right) &&= \left(a + c; b + d\right) \\
                &\left(a; b\right) - \left(c; d\right) &&= \left(a - c; b - d\right) \\
                &\left(a; b\right) \cdot \left(c; d\right) &&= \left(ac - bd; ad + bc\right) \\
            \end{alignat*}
        }
    \end{tabular}
}

\subsection{Komplexe Zahlen als Vektorraum}

\parbox{0.9\textwidth}
{
    Eine komplexe Zahl als Wertepaar kann mit einem Vektor im Vektorraum $\mathbb{R}^2$ verglichen werden,
    wobei auch Rechenarten, wie Addition und Multiplikation analog zur Vektoraddition und Vektormultiplikation sind.
    Der Vektorraum bei komplexen Zahlen hat wie der Vektorraum von $\mathbb{R}^2$ zwei Koordinatenachsen.
    Dabei gibt die x-Achse den reellen Anteil und die y-Achse den imaginären Anteil der komplexen Zahl an.
    Für die Darstellung komplexer Zahlen wird die Gleichung 
    $\mathbf{z = \left(a; b\right) = a \cdot \left(1; 0\right) + b \cdot \left(0;1\right)}$ verwendet.
    Dabei ist das Wertepaar $\mathbf{\left(1; 0\right) = 1}$ als Einselement definiert, während das Wertepaar 
    $\mathbf{\left(0; 1\right) = i}$ angibt. Es ergibt sich also folgende Gleichung: 
    $\mathbf{z = a \cdot 1 + b \cdot i = a + bi}$.
}

\subsubsection{Arithmetik}

\begin{alignat*}{5}
    &z_1 + z_2     &&= \left(a_1; b_1\right) + \left(a_2; b_2\right)     &&= \left(a_1 + a_2; b_1 + b_2\right) \\
    &z_1 - z_2     &&= \left(a_1; b_1\right) - \left(a_2; b_2\right)     &&= \left(a_1 - a_2; b_1 - b_2\right) \\
    &z_1 \cdot z_2 &&= \left(a_1; b_1\right) \cdot \left(a_2; b_2\right) &&= \left(a_1 \cdot a_2 - b_1 \cdot b_2; a_1 \cdot b_2 + a_2 \cdot b_1\right)
\end{alignat*}

\resizebox{10cm}{10cm}{
    \begin{tikzpicture}
        \begin{axis}[
            samples=100,
            axis lines=center,
            x=1cm,
            y=1cm,
            xmin=-5, xmax=5,
            ymin=-5, ymax=5,
            xtick={-4, -3,...,3, 4},
            ytick={-4, -3,...,3, 4},
            xticklabels={$-4$,$-3$,$-2$,$-1$,$0$,$1$,$2$,$3$,$4$},
            yticklabels={$-4i$,$-3i$,$-2i$,$-1i$,$0i$,$1i$,$2i$,$3i$,$4i$},
            xlabel=$\mbox{Re}$,
            ylabel=$\mbox{Im}$
        ]

        \draw[fill] (2, 3) circle[radius=2pt, fill] node[above right]{ z };

        \draw[->] (0, 0) -- (2, 3) node[above right]{ z };
        
        \draw[decorate, decoration = {brace,amplitude=6pt,raise=1pt}] (0, 3) -- (2, 3) node[midway, above, yshift=5pt]{a};
        \draw[decorate, decoration = {brace,amplitude=6pt,raise=1pt,mirror}] (2, 0) -- (2, 3) node[midway, right, xshift=5pt]{b};

        \end{axis}
    \end{tikzpicture}
}

