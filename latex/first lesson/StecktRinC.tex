\parbox{0.9\textwidth}
{
    Nun haben wir eine neue Menge, in der $\mathbf{\sqrt{-1} = i}$ ist, jedoch
    wollen wir nun beweisen, dass die Menge eine \textbf{Erweiterung der reellen Zahlen} $\mathbb{R}$ ist.
    Dazu nutzen wir unsere vorherige Definition, laut der $\mathbf{x \cdot 1 = \left(x; 0\right)}$ ist,
    nun nehmen wir zwei Zahlen $\mathbf{a}$, $\mathbf{b}$ mit $\mathbf{a, b \in \mathbb{R}}$, schreiben
    sie nach Definition auf: $\mathbf{a \cdot 1 = \left(a; 0\right)}$ und $\mathbf{b \cdot 1 = \left(b; 0\right)}$
    und addieren sie $\mathbf{a \cdot 1 + b \cdot 1 = \left(a + b\right) \cdot 1}$ und
    $\mathbf{\left(a; 0\right) + \left(b; 0\right) = \left(a + b; 0\right) = \left(a + b\right) \cdot 1}$
    beim Vergleichen, stellen wir fest, dass die beiden Ergebnisse \textbf{gleich} sind, dass bedeutet,
    dass alle Elemente aus $\mathbf{\mathbb{C}}$ mit $\mathbf{(a \cdot 1) \in \mathbb{R}}$ sich \textbf{gleich wie
    die Reellen Zahlen verhalten}. \\

    Wir schreiben also $z=(a;b)=a\mathbf{1}+b\mathbf{i}$, jedoch verzichten wir auf
    die $\mathbf{1}$ und schreiben $\mathbf{i}$ ohne Fettdruck, also $z = (a;b) = a + b \cdot i$.
}